%% -*- latex -*-

% Presentation
%\documentclass[aspectratio=1610]{beamer} % Macbook Pro screen 16:10
%% \documentclass{beamer} % default aspect ratio 4:3
\documentclass[handout]{beamer}

% \setbeameroption{show notes} % un-comment to see the notes

%% To do: trim these definitions

\newcommand\nc\newcommand
\nc\rnc\renewcommand

\usepackage{epsfig}
\usepackage{latexsym}

%% %% lhs2tex messes with verbatim, and I lose spacing.
%% \usepackage{fancyvrb}

%% \usepackage{enumitem}
%% \setlist[1]{labelindent=\parindent,itemsep=0.5in,label=$\bullet$}
%% \setlist[2]{labelindent=\parindent,itemsep=0.5in,label=$\bullet$}

\nc\out[1]{}

\nc\mynoteOut[2]{\mynote{#1}\out{#2}}

% While working, use these defs
%% \nc\mynote[1]{{\em [#1]}}
%% \nc\mynotefoot[1]{\footnote{\mynote{#1}}}
% But for the submission, use these
\nc\mynote\out
\nc\mynotefoot\out

\nc\todo{\mynote{To do.}}

\nc\figlabel[1]{\label{fig:#1}}
\nc\figref[1]{Figure~\ref{fig:#1}}

\nc\needcite{\mynote{ref}}

% \nc{\Opid}[1]{\operatorname{#1}}
\nc{\Opid}[1]{\Varid{#1}}
\nc{\Varap}[1]{\Opid{#1}\,}
\nc{\Varapp}[2]{\Varap{#1}{(#2)}}

\nc\wpicture[2]{\includegraphics[width=#1]{Figures/#2}}

\nc\wfig[2]{
\begin{center}
\wpicture{#1}{#2}
\end{center}
}
\nc\fig[1]{\wfig{4in}{#1}}

\nc\usebg[1]{\usebackgroundtemplate{\wpicture{1.2\textwidth}{#1}}}

\nc\framet[2]{\frame{\frametitle{#1}#2}}

\nc\hidden[1]{}


\newcommand{\Pair}{\Varid{Pair}}

\newcommand{\stat}[6]{
#1 & #2 & #3 & #4 & #5 & #6 \\ \hline
}
\newcommand{\fftStats}[1]{
\begin{center}
\begin{tabular}{|c|c|c|c|c|c|}
  \hline
  \stat{Type}{$+$}{$\times$}{$-$}{total}{max depth} \hline
  #1
  \hline
\end{tabular}
\end{center}
}

\nc{\subo}{_{\!\mathit{1}}}

\nc\partframe[1]{\framet{}{\begin{center} \vspace{6ex} {\Huge \textcolor{partColor}{#1}} \end{center}}}

\nc\symTwo[1]{\mathbin{#1\!\!\!#1}}
\nc\symThree[1]{\mathbin{#1\!\!#1\!\!#1}}


\usefonttheme{serif}
\usepackage{framed}
\usepackage{hyperref}
\usepackage{color}

\definecolor{linkColor}{rgb}{0,0.42,0.3}
\definecolor{partColor}{rgb}{0,0,0.8}

\hypersetup{colorlinks=true,urlcolor=linkColor}

\usepackage{graphicx}
\usepackage{color}
\DeclareGraphicsExtensions{.pdf,.png,.jpg}

\usepackage{geometry}
% \usepackage[a4paper]{geometry}

%% \usepackage{wasysym}
\usepackage{mathabx}
\usepackage{setspace}
\usepackage{enumerate}
\usepackage{tikzsymbols}
% \usepackage{fancybox}
\usepackage[many]{tcolorbox}

\tcbset{enhanced,boxrule=0.5pt,colframe=black!50!blue,colback=white,boxsep=-2pt,drop fuzzy shadow}

\usepackage[absolute,overlay]{textpos}  % ,showboxes

\TPGrid{364}{273} %% roughly page size in points

\useinnertheme[shadow]{rounded}
% \useoutertheme{default}
\useoutertheme{shadow}
\useoutertheme{infolines}
% Suppress navigation arrows
\setbeamertemplate{navigation symbols}{}

\newcommand\sourced[1]{\href{#1}{\tiny (source)}}


\definecolor{statColor}{rgb}{0,0.5,0}

\newcommand{\stats}[2]{
{\small \textcolor{statColor}{work: #1, depth: #2}}}

\newcommand\ccircuit[3]{
\framet{#1}{
\vspace{#2ex}
\wfig{4.5in}{circuits/#3}
}}

\newcommand\circuit[5]{
\ccircuit{#1 \hfill \stats {#4}{#5}\hspace{2ex}}{#2}{#3}
}

\DeclareMathOperator{\D}{Depth}
\DeclareMathOperator{\W}{Work}
\nc\Size[1]{\lvert #1 \rvert}


\author{\href{http://conal.net}{Conal Elliott}}
%% \institute{Target}

\graphicspath{{Figures/}}

\definecolor{shadecolor}{rgb}{0.95,0.95,0.95}
\setlength{\fboxsep}{0.75ex}
\setlength{\fboxrule}{0.15pt}
%% \setlength{\shadowsize}{2pt}

%% \nc\cbox[1]{\raisebox{-0.5\height}{\fbox{#1}}}
\nc\cpic[2]{\fbox{\wpicture{#1}{circuits/#2}}}
\nc\ccap[3]{
\begin{minipage}[c]{0.48\textwidth}
\begin{center}
\cpic{#2}{#3}\par\vspace{0.5ex}#1\par
\end{center}
\end{minipage}
}

\setlength{\itemsep}{2ex}
\setlength{\parskip}{1ex}
% \setstretch{1.2} % ??

\nc\pitem{\pause\item}

%% Double quote symbol
\nc\dq{\text{\tt\char34}}
%% Quoted haskell string with formatted content
\nc\hquoted[1]{\dq\!#1\!\dq}

%% Pandoc introduces tightlist
\nc\tightlist{\itemsep1ex}

\nc\btimes{\pmb{\times}}
\nc\bplus{\pmb{+}}
% \nc\bcomp{\pmb{\circ}}
\nc\bcomp{\circ}


%% ODER: format ==         = "\mathrel{==}"
%% ODER: format /=         = "\neq "
%
%
\makeatletter
\@ifundefined{lhs2tex.lhs2tex.sty.read}%
  {\@namedef{lhs2tex.lhs2tex.sty.read}{}%
   \newcommand\SkipToFmtEnd{}%
   \newcommand\EndFmtInput{}%
   \long\def\SkipToFmtEnd#1\EndFmtInput{}%
  }\SkipToFmtEnd

\newcommand\ReadOnlyOnce[1]{\@ifundefined{#1}{\@namedef{#1}{}}\SkipToFmtEnd}
\usepackage{amstext}
\usepackage{amssymb}
\usepackage{stmaryrd}
\DeclareFontFamily{OT1}{cmtex}{}
\DeclareFontShape{OT1}{cmtex}{m}{n}
  {<5><6><7><8>cmtex8
   <9>cmtex9
   <10><10.95><12><14.4><17.28><20.74><24.88>cmtex10}{}
\DeclareFontShape{OT1}{cmtex}{m}{it}
  {<-> ssub * cmtt/m/it}{}
\newcommand{\texfamily}{\fontfamily{cmtex}\selectfont}
\DeclareFontShape{OT1}{cmtt}{bx}{n}
  {<5><6><7><8>cmtt8
   <9>cmbtt9
   <10><10.95><12><14.4><17.28><20.74><24.88>cmbtt10}{}
\DeclareFontShape{OT1}{cmtex}{bx}{n}
  {<-> ssub * cmtt/bx/n}{}
\newcommand{\tex}[1]{\text{\texfamily#1}}	% NEU

\newcommand{\Sp}{\hskip.33334em\relax}


\newcommand{\Conid}[1]{\mathit{#1}}
\newcommand{\Varid}[1]{\mathit{#1}}
\newcommand{\anonymous}{\kern0.06em \vbox{\hrule\@width.5em}}
\newcommand{\plus}{\mathbin{+\!\!\!+}}
\newcommand{\bind}{\mathbin{>\!\!\!>\mkern-6.7mu=}}
\newcommand{\rbind}{\mathbin{=\mkern-6.7mu<\!\!\!<}}% suggested by Neil Mitchell
\newcommand{\sequ}{\mathbin{>\!\!\!>}}
\renewcommand{\leq}{\leqslant}
\renewcommand{\geq}{\geqslant}
\usepackage{polytable}

%mathindent has to be defined
\@ifundefined{mathindent}%
  {\newdimen\mathindent\mathindent\leftmargini}%
  {}%

\def\resethooks{%
  \global\let\SaveRestoreHook\empty
  \global\let\ColumnHook\empty}
\newcommand*{\savecolumns}[1][default]%
  {\g@addto@macro\SaveRestoreHook{\savecolumns[#1]}}
\newcommand*{\restorecolumns}[1][default]%
  {\g@addto@macro\SaveRestoreHook{\restorecolumns[#1]}}
\newcommand*{\aligncolumn}[2]%
  {\g@addto@macro\ColumnHook{\column{#1}{#2}}}

\resethooks

\newcommand{\onelinecommentchars}{\quad-{}- }
\newcommand{\commentbeginchars}{\enskip\{-}
\newcommand{\commentendchars}{-\}\enskip}

\newcommand{\visiblecomments}{%
  \let\onelinecomment=\onelinecommentchars
  \let\commentbegin=\commentbeginchars
  \let\commentend=\commentendchars}

\newcommand{\invisiblecomments}{%
  \let\onelinecomment=\empty
  \let\commentbegin=\empty
  \let\commentend=\empty}

\visiblecomments

\newlength{\blanklineskip}
\setlength{\blanklineskip}{0.66084ex}

\newcommand{\hsindent}[1]{\quad}% default is fixed indentation
\let\hspre\empty
\let\hspost\empty
\newcommand{\NB}{\textbf{NB}}
\newcommand{\Todo}[1]{$\langle$\textbf{To do:}~#1$\rangle$}

\EndFmtInput
\makeatother
%
%
%
%
%
%
% This package provides two environments suitable to take the place
% of hscode, called "plainhscode" and "arrayhscode". 
%
% The plain environment surrounds each code block by vertical space,
% and it uses \abovedisplayskip and \belowdisplayskip to get spacing
% similar to formulas. Note that if these dimensions are changed,
% the spacing around displayed math formulas changes as well.
% All code is indented using \leftskip.
%
% Changed 19.08.2004 to reflect changes in colorcode. Should work with
% CodeGroup.sty.
%
\ReadOnlyOnce{polycode.fmt}%
\makeatletter

\newcommand{\hsnewpar}[1]%
  {{\parskip=0pt\parindent=0pt\par\vskip #1\noindent}}

% can be used, for instance, to redefine the code size, by setting the
% command to \small or something alike
\newcommand{\hscodestyle}{}

% The command \sethscode can be used to switch the code formatting
% behaviour by mapping the hscode environment in the subst directive
% to a new LaTeX environment.

\newcommand{\sethscode}[1]%
  {\expandafter\let\expandafter\hscode\csname #1\endcsname
   \expandafter\let\expandafter\endhscode\csname end#1\endcsname}

% "compatibility" mode restores the non-polycode.fmt layout.

\newenvironment{compathscode}%
  {\par\noindent
   \advance\leftskip\mathindent
   \hscodestyle
   \let\\=\@normalcr
   \let\hspre\(\let\hspost\)%
   \pboxed}%
  {\endpboxed\)%
   \par\noindent
   \ignorespacesafterend}

\newcommand{\compaths}{\sethscode{compathscode}}

% "plain" mode is the proposed default.
% It should now work with \centering.
% This required some changes. The old version
% is still available for reference as oldplainhscode.

\newenvironment{plainhscode}%
  {\hsnewpar\abovedisplayskip
   \advance\leftskip\mathindent
   \hscodestyle
   \let\hspre\(\let\hspost\)%
   \pboxed}%
  {\endpboxed%
   \hsnewpar\belowdisplayskip
   \ignorespacesafterend}

\newenvironment{oldplainhscode}%
  {\hsnewpar\abovedisplayskip
   \advance\leftskip\mathindent
   \hscodestyle
   \let\\=\@normalcr
   \(\pboxed}%
  {\endpboxed\)%
   \hsnewpar\belowdisplayskip
   \ignorespacesafterend}

% Here, we make plainhscode the default environment.

\newcommand{\plainhs}{\sethscode{plainhscode}}
\newcommand{\oldplainhs}{\sethscode{oldplainhscode}}
\plainhs

% The arrayhscode is like plain, but makes use of polytable's
% parray environment which disallows page breaks in code blocks.

\newenvironment{arrayhscode}%
  {\hsnewpar\abovedisplayskip
   \advance\leftskip\mathindent
   \hscodestyle
   \let\\=\@normalcr
   \(\parray}%
  {\endparray\)%
   \hsnewpar\belowdisplayskip
   \ignorespacesafterend}

\newcommand{\arrayhs}{\sethscode{arrayhscode}}

% The mathhscode environment also makes use of polytable's parray 
% environment. It is supposed to be used only inside math mode 
% (I used it to typeset the type rules in my thesis).

\newenvironment{mathhscode}%
  {\parray}{\endparray}

\newcommand{\mathhs}{\sethscode{mathhscode}}

% texths is similar to mathhs, but works in text mode.

\newenvironment{texthscode}%
  {\(\parray}{\endparray\)}

\newcommand{\texths}{\sethscode{texthscode}}

% The framed environment places code in a framed box.

\def\codeframewidth{\arrayrulewidth}
\RequirePackage{calc}

\newenvironment{framedhscode}%
  {\parskip=\abovedisplayskip\par\noindent
   \hscodestyle
   \arrayrulewidth=\codeframewidth
   \tabular{@{}|p{\linewidth-2\arraycolsep-2\arrayrulewidth-2pt}|@{}}%
   \hline\framedhslinecorrect\\{-1.5ex}%
   \let\endoflinesave=\\
   \let\\=\@normalcr
   \(\pboxed}%
  {\endpboxed\)%
   \framedhslinecorrect\endoflinesave{.5ex}\hline
   \endtabular
   \parskip=\belowdisplayskip\par\noindent
   \ignorespacesafterend}

\newcommand{\framedhslinecorrect}[2]%
  {#1[#2]}

\newcommand{\framedhs}{\sethscode{framedhscode}}

% The inlinehscode environment is an experimental environment
% that can be used to typeset displayed code inline.

\newenvironment{inlinehscode}%
  {\(\def\column##1##2{}%
   \let\>\undefined\let\<\undefined\let\\\undefined
   \newcommand\>[1][]{}\newcommand\<[1][]{}\newcommand\\[1][]{}%
   \def\fromto##1##2##3{##3}%
   \def\nextline{}}{\) }%

\newcommand{\inlinehs}{\sethscode{inlinehscode}}

% The joincode environment is a separate environment that
% can be used to surround and thereby connect multiple code
% blocks.

\newenvironment{joincode}%
  {\let\orighscode=\hscode
   \let\origendhscode=\endhscode
   \def\endhscode{\def\hscode{\endgroup\def\@currenvir{hscode}\\}\begingroup}
   %\let\SaveRestoreHook=\empty
   %\let\ColumnHook=\empty
   %\let\resethooks=\empty
   \orighscode\def\hscode{\endgroup\def\@currenvir{hscode}}}%
  {\origendhscode
   \global\let\hscode=\orighscode
   \global\let\endhscode=\origendhscode}%

\makeatother
\EndFmtInput
%
%
%
% First, let's redefine the forall, and the dot.
%
%
% This is made in such a way that after a forall, the next
% dot will be printed as a period, otherwise the formatting
% of `comp_` is used. By redefining `comp_`, as suitable
% composition operator can be chosen. Similarly, period_
% is used for the period.
%
\ReadOnlyOnce{forall.fmt}%
\makeatletter

% The HaskellResetHook is a list to which things can
% be added that reset the Haskell state to the beginning.
% This is to recover from states where the hacked intelligence
% is not sufficient.

\let\HaskellResetHook\empty
\newcommand*{\AtHaskellReset}[1]{%
  \g@addto@macro\HaskellResetHook{#1}}
\newcommand*{\HaskellReset}{\HaskellResetHook}

\global\let\hsforallread\empty

\newcommand\hsforall{\global\let\hsdot=\hsperiodonce}
\newcommand*\hsperiodonce[2]{#2\global\let\hsdot=\hscompose}
\newcommand*\hscompose[2]{#1}

\AtHaskellReset{\global\let\hsdot=\hscompose}

% In the beginning, we should reset Haskell once.
\HaskellReset

\makeatother
\EndFmtInput
%
% -*- text -*-
%% Misc lhs2TeX directives


%% spaces (in 18ths of a quad): \, = 3, \: = 4, \; = 5, \! = -3



%% optional double-dollar spelling, to avoid $ confusing emacs latex-mode.


%% %format <*> = "\circledast"

%% hack: add missing space, e.g., before "{" in data type decl











%% %format <.> = "\mathbin{<\!\!\! \cdot \!\!\!>}"

%% %format $@ = "\mathbin{\hat\$}"
%% %% %format $@ = "\mathbin{\$\!@}"





\title[]{The simple essence of automatic differentiation}
\date{January 2018}
\institute[]{Target}

\setlength{\itemsep}{2ex}
\setlength{\parskip}{1ex}
\setlength{\blanklineskip}{1.5ex}
\setlength\mathindent{4ex}

\nc\wow\emph

\begin{document}

% \large

\frame{\titlepage}
\title{The simple essence of automatic differentiation}
\institute{Target}

\framet{What's a derivative?}{

For scalar domain: $$\ensuremath{\Varid{der}\;\Varid{f}\;\Varid{x}\mathrel{=}\lim_{\varepsilon \to \mathrm{0}}{\frac{\Varid{f}\;(\Varid{x}\mathbin{+}\varepsilon )\mathbin{-}\Varid{f}\;\Varid{x}}{\varepsilon }}}$$

\ 
\pause
Redefine: unique scalar $s$ such that
 $$ \ensuremath{\lim_{\varepsilon \to \mathrm{0}}{\frac{\Varid{f}\;(\Varid{x}\mathbin{+}\varepsilon )\mathbin{-}\Varid{f}\;\Varid{x}}{\varepsilon }}\mathbin{-}\Varid{s}\equiv \mathrm{0}} $$

\pause
Equivalently,
 $$ \ensuremath{\lim_{\varepsilon \to \mathrm{0}}{\frac{\Varid{f}\;(\Varid{x}\mathbin{+}\varepsilon )\mathbin{-}(\Varid{f}\;\Varid{x}\mathbin{+}\Varid{s} \cdot \varepsilon )}{\varepsilon }}\equiv \mathrm{0}} $$

}

\framet{What's a derivative?}{
For scalar domain:
 $$ \ensuremath{\lim_{\varepsilon \to \mathrm{0}}{\frac{\Varid{f}\;(\Varid{x}\mathbin{+}\varepsilon )\mathbin{-}(\Varid{f}\;\Varid{x}\mathbin{+}\Varid{s} \cdot \varepsilon )}{\varepsilon }}\equiv \mathrm{0}} $$

\pause

Now generalize: unique \wow{linear map} $T$ such that

$$\ensuremath{\lim_{\varepsilon \to \mathrm{0}}{\frac{|\Varid{f}\;(\Varid{x}\mathbin{+}\varepsilon )\mathbin{-}(\Varid{f}\;\Varid{x}\mathbin{+}\Conid{T}\;\varepsilon )|}{|\varepsilon |}}\equiv \mathrm{0}}$$

\pause\vspace{3ex}

\wow{Derivatives are linear maps:}
\begin{hscode}\SaveRestoreHook
\column{B}{@{}>{\hspre}l<{\hspost}@{}}%
\column{3}{@{}>{\hspre}l<{\hspost}@{}}%
\column{E}{@{}>{\hspre}l<{\hspost}@{}}%
\>[3]{}\Varid{der}\mathbin{::}(\Varid{a}\to \Varid{b})\to (\Varid{a}\to (\Varid{a}\multimap\Varid{b})){}\<[E]%
\ColumnHook
\end{hscode}\resethooks
%% Captures all ``partial derivatives'' for all dimensions.

See \emph{Calculus on Manifolds} by Michael Spivak.

}

\framet{Composition}{

Sequential and parallel:

\begin{hscode}\SaveRestoreHook
\column{B}{@{}>{\hspre}l<{\hspost}@{}}%
\column{E}{@{}>{\hspre}l<{\hspost}@{}}%
\>[B]{}(\hsdot{\circ }{.})\mathbin{::}(\Varid{b}\to \Varid{c})\to (\Varid{a}\to \Varid{b})\to (\Varid{a}\to \Varid{c}){}\<[E]%
\\
\>[B]{}(\Varid{g}\hsdot{\circ }{.}\Varid{f})\;\Varid{a}\mathrel{=}\Varid{g}\;(\Varid{f}\;\Varid{a}){}\<[E]%
\\
\>[B]{}{}{}\<[E]%
\\
\>[B]{}(\mathbin{\smalltriangleup})\mathbin{::}(\Varid{a}\to \Varid{c})\to (\Varid{a}\to \Varid{d})\to (\Varid{a}\to \Varid{c} \times \Varid{d}){}\<[E]%
\\
\>[B]{}(\Varid{f}\mathbin{\smalltriangleup}\Varid{g})\;\Varid{a}\mathrel{=}(\Varid{f}\;\Varid{a},\Varid{g}\;\Varid{a}){}\<[E]%
\ColumnHook
\end{hscode}\resethooks

\pause
Differentiation rules:
\begin{hscode}\SaveRestoreHook
\column{B}{@{}>{\hspre}l<{\hspost}@{}}%
\column{43}{@{}>{\hspre}l<{\hspost}@{}}%
\column{E}{@{}>{\hspre}l<{\hspost}@{}}%
\>[B]{}\Varid{der}\;(\Varid{g}\hsdot{\circ }{.}\Varid{f})\;\Varid{a}\equiv \Varid{der}\;\Varid{g}\;(\Varid{f}\;\Varid{a})\hsdot{\circ }{.}\Varid{der}\;\Varid{f}\;\Varid{a}{}\<[43]%
\>[43]{}\mbox{\onelinecomment  ``chain rule''}{}\<[E]%
\\
\>[B]{}{}{}\<[E]%
\\
\>[B]{}\Varid{der}\;(\Varid{g}\mathbin{\smalltriangleup}\Varid{f})\;\Varid{a}\equiv \Varid{der}\;\Varid{g}\;\Varid{a}\mathbin{\smalltriangleup}\Varid{der}\;\Varid{f}\;\Varid{a}{}\<[E]%
\ColumnHook
\end{hscode}\resethooks
}

\framet{Compositionality}{

Chain rule:
\begin{hscode}\SaveRestoreHook
\column{B}{@{}>{\hspre}l<{\hspost}@{}}%
\column{3}{@{}>{\hspre}l<{\hspost}@{}}%
\column{46}{@{}>{\hspre}l<{\hspost}@{}}%
\column{E}{@{}>{\hspre}l<{\hspost}@{}}%
\>[3]{}\Varid{der}\;(\Varid{g}\hsdot{\circ }{.}\Varid{f})\;\Varid{a}\equiv \Varid{der}\;\Varid{g}\;(\Varid{f}\;\Varid{a})\hsdot{\circ }{.}\Varid{der}\;\Varid{f}\;\Varid{a}{}\<[46]%
\>[46]{}\mbox{\onelinecomment  non-compositional}{}\<[E]%
\ColumnHook
\end{hscode}\resethooks
\ 

To fix, combine primal (regular result) with derivative:
\begin{hscode}\SaveRestoreHook
\column{B}{@{}>{\hspre}l<{\hspost}@{}}%
\column{28}{@{}>{\hspre}l<{\hspost}@{}}%
\column{E}{@{}>{\hspre}l<{\hspost}@{}}%
\>[B]{}\Varid{andDer}\mathbin{::}(\Varid{a}\to \Varid{b})\to (\Varid{a}\to (\Varid{b} \times (\Varid{a}\multimap\Varid{b}))){}\<[E]%
\\
\>[B]{}\Varid{andDer}\;\Varid{f}\mathrel{=}\Varid{f}\mathbin{\smalltriangleup}\Varid{der}\;\Varid{f}{}\<[28]%
\>[28]{}\mbox{\onelinecomment  specification}{}\<[E]%
\ColumnHook
\end{hscode}\resethooks

Often much work in common to \ensuremath{\Varid{f}} and \ensuremath{\Varid{der}\;\Varid{f}}.
}

\framet{Linear functions are their own derivatives everywhere.}{

Linear functions are their own perfect linear approximations.

\begin{hscode}\SaveRestoreHook
\column{B}{@{}>{\hspre}l<{\hspost}@{}}%
\column{10}{@{}>{\hspre}l<{\hspost}@{}}%
\column{13}{@{}>{\hspre}c<{\hspost}@{}}%
\column{13E}{@{}l@{}}%
\column{16}{@{}>{\hspre}l<{\hspost}@{}}%
\column{E}{@{}>{\hspre}l<{\hspost}@{}}%
\>[B]{}\Varid{der}\;\Varid{id}\;{}\<[10]%
\>[10]{}\Varid{a}{}\<[13]%
\>[13]{}\mathrel{=}{}\<[13E]%
\>[16]{}\Varid{id}{}\<[E]%
\\
\>[B]{}\Varid{der}\;\Varid{fst}\;{}\<[10]%
\>[10]{}\Varid{a}{}\<[13]%
\>[13]{}\mathrel{=}{}\<[13E]%
\>[16]{}\Varid{fst}{}\<[E]%
\\
\>[B]{}\Varid{der}\;\Varid{snd}\;{}\<[10]%
\>[10]{}\Varid{a}{}\<[13]%
\>[13]{}\mathrel{=}{}\<[13E]%
\>[16]{}\Varid{snd}{}\<[E]%
\\
\>[13]{}\mathbin{...}{}\<[13E]%
\ColumnHook
\end{hscode}\resethooks

For linear functions \ensuremath{\Varid{f}},
\begin{hscode}\SaveRestoreHook
\column{B}{@{}>{\hspre}l<{\hspost}@{}}%
\column{3}{@{}>{\hspre}l<{\hspost}@{}}%
\column{E}{@{}>{\hspre}l<{\hspost}@{}}%
\>[3]{}\Varid{andDer}\;\Varid{f}\;\Varid{a}\mathrel{=}(\Varid{f}\;\Varid{a},\Varid{f}){}\<[E]%
\ColumnHook
\end{hscode}\resethooks
%% > andDer f = f &&& const f

}

%%  %format da = "\Delta a"
%%  %format db = "\Delta b"

\framet{Abstract algebra for functions}{

\begin{hscode}\SaveRestoreHook
\column{B}{@{}>{\hspre}l<{\hspost}@{}}%
\column{3}{@{}>{\hspre}l<{\hspost}@{}}%
\column{8}{@{}>{\hspre}l<{\hspost}@{}}%
\column{10}{@{}>{\hspre}c<{\hspost}@{}}%
\column{10E}{@{}l@{}}%
\column{14}{@{}>{\hspre}l<{\hspost}@{}}%
\column{25}{@{}>{\hspre}l<{\hspost}@{}}%
\column{39}{@{}>{\hspre}l<{\hspost}@{}}%
\column{E}{@{}>{\hspre}l<{\hspost}@{}}%
\>[B]{}\mathbf{class}\;\Conid{Category}\;\Varid{k}\;\mathbf{where}{}\<[E]%
\\
\>[B]{}\hsindent{3}{}\<[3]%
\>[3]{}\Varid{id}{}\<[8]%
\>[8]{}\mathbin{::}\Varid{a}\mathbin{`\Varid{k}`}\Varid{a}{}\<[E]%
\\
\>[B]{}\hsindent{3}{}\<[3]%
\>[3]{}(\hsdot{\circ }{.}){}\<[8]%
\>[8]{}\mathbin{::}(\Varid{b}\mathbin{`\Varid{k}`}\Varid{c})\to (\Varid{a}\mathbin{`\Varid{k}`}\Varid{b})\to (\Varid{a}\mathbin{`\Varid{k}`}\Varid{c}){}\<[E]%
\\
\>[B]{}\hsindent{3}{}\<[3]%
\>[3]{}\mathbf{infixr}\;\mathrm{9}\hsdot{\circ }{.}{}\<[E]%
\\[\blanklineskip]%
\>[B]{}{}{}\<[E]%
\\[\blanklineskip]%
\>[B]{}\mathbf{class}\;\Conid{Category}\;\Varid{k}\Rightarrow \Conid{ProductCat}\;\Varid{k}\;\mathbf{where}{}\<[E]%
\\
\>[B]{}\hsindent{3}{}\<[3]%
\>[3]{}\mathbf{type}\;\Conid{Prod}\;\Varid{k}\;\Varid{a}\;\Varid{b}{}\<[E]%
\\
\>[B]{}\hsindent{3}{}\<[3]%
\>[3]{}\Varid{exl}{}\<[10]%
\>[10]{}\mathbin{::}{}\<[10E]%
\>[14]{}(\Conid{Prod}\;\Varid{k}\;\Varid{a}\;\Varid{b})\mathbin{`\Varid{k}`}\Varid{a}{}\<[E]%
\\
\>[B]{}\hsindent{3}{}\<[3]%
\>[3]{}\Varid{exr}{}\<[10]%
\>[10]{}\mathbin{::}{}\<[10E]%
\>[14]{}(\Conid{Prod}\;\Varid{k}\;\Varid{a}\;\Varid{b})\mathbin{`\Varid{k}`}\Varid{b}{}\<[E]%
\\
\>[B]{}\hsindent{3}{}\<[3]%
\>[3]{}(\mathbin{\smalltriangleup}){}\<[10]%
\>[10]{}\mathbin{::}{}\<[10E]%
\>[14]{}(\Varid{a}\mathbin{`\Varid{k}`}\Varid{c}){}\<[25]%
\>[25]{}\to (\Varid{a}\mathbin{`\Varid{k}`}\Varid{d}){}\<[39]%
\>[39]{}\to (\Varid{a}\mathbin{`\Varid{k}`}(\Conid{Prod}\;\Varid{k}\;\Varid{c}\;\Varid{d})){}\<[E]%
\\
\>[B]{}\hsindent{3}{}\<[3]%
\>[3]{}\mathbf{infixr}\;\mathrm{3}\mathbin{\smalltriangleup}{}\<[E]%
\ColumnHook
\end{hscode}\resethooks

Plus laws and classes for arithmetic etc.

}

\framet{A simple AD implementation}{
\mathindent-1ex
\begin{hscode}\SaveRestoreHook
\column{B}{@{}>{\hspre}l<{\hspost}@{}}%
\column{E}{@{}>{\hspre}l<{\hspost}@{}}%
\>[B]{}\mathbf{data}\;\Conid{D}\;\Varid{a}\;\Varid{b}\mathrel{=}\Conid{D}\;(\Varid{a}\to \Varid{b} \times (\Varid{a}\multimap\Varid{b}))\mbox{\onelinecomment  Derivatives are linear maps.}{}\<[E]%
\ColumnHook
\end{hscode}\resethooks
\pause
\vspace{-4ex}
\begin{hscode}\SaveRestoreHook
\column{B}{@{}>{\hspre}l<{\hspost}@{}}%
\column{3}{@{}>{\hspre}l<{\hspost}@{}}%
\column{8}{@{}>{\hspre}l<{\hspost}@{}}%
\column{9}{@{}>{\hspre}l<{\hspost}@{}}%
\column{E}{@{}>{\hspre}l<{\hspost}@{}}%
\>[B]{}\Varid{linearD}\;\Varid{f}\mathrel{=}\Conid{D}\;(\Varid{f}\mathbin{\smalltriangleup}\Varid{const}\;\Varid{f}){}\<[E]%
\\[\blanklineskip]%
\>[B]{}\mathbf{instance}\;\Conid{Category}\;\Conid{D}\;\mathbf{where}{}\<[E]%
\\
\>[B]{}\hsindent{3}{}\<[3]%
\>[3]{}\Varid{id}\mathrel{=}\Varid{linearD}\;\Varid{id}{}\<[E]%
\\
\>[B]{}\hsindent{3}{}\<[3]%
\>[3]{}\Conid{D}\;\Varid{g}\hsdot{\circ }{.}\Conid{D}\;\Varid{f}\mathrel{=}\Conid{D}\;(\lambda \Varid{a}\to \mathbf{let}\;\{\mskip1.5mu (\Varid{b},\Varid{f'})\mathrel{=}\Varid{f}\;\Varid{a};(\Varid{c},\Varid{g'})\mathrel{=}\Varid{g}\;\Varid{b}\mskip1.5mu\}\;\mathbf{in}\;(\Varid{c},\Varid{g'}\hsdot{\circ }{.}\Varid{f'})){}\<[E]%
\\[\blanklineskip]%
\>[B]{}\mathbf{instance}\;\Conid{Cartesian}\;\Conid{D}\;\mathbf{where}{}\<[E]%
\\
\>[B]{}\hsindent{3}{}\<[3]%
\>[3]{}\Varid{exl}{}\<[8]%
\>[8]{}\mathrel{=}\Varid{linearD}\;\Varid{exl}{}\<[E]%
\\
\>[B]{}\hsindent{3}{}\<[3]%
\>[3]{}\Varid{exr}{}\<[8]%
\>[8]{}\mathrel{=}\Varid{linearD}\;\Varid{exr}{}\<[E]%
\\
\>[B]{}\hsindent{3}{}\<[3]%
\>[3]{}\Conid{D}\;\Varid{f}\mathbin{\smalltriangleup}\Conid{D}\;\Varid{g}\mathrel{=}\Conid{D}\;(\lambda \Varid{a}\to \mathbf{let}\;\{\mskip1.5mu (\Varid{b},\Varid{f'})\mathrel{=}\Varid{f}\;\Varid{a};(\Varid{c},\Varid{g'})\mathrel{=}\Varid{g}\;\Varid{a}\mskip1.5mu\}\;\mathbf{in}\;((\Varid{b},\Varid{c}),\Varid{f'}\mathbin{\smalltriangleup}\Varid{g'})){}\<[E]%
\\[\blanklineskip]%
\>[B]{}\mathbf{instance}\;\Conid{NumCat}\;\Conid{D}\;\mathbf{where}{}\<[E]%
\\
\>[B]{}\hsindent{3}{}\<[3]%
\>[3]{}\Varid{negateC}\mathrel{=}\Varid{linearD}\;\Varid{negateC}{}\<[E]%
\\
\>[B]{}\hsindent{3}{}\<[3]%
\>[3]{}\Varid{addC}{}\<[9]%
\>[9]{}\mathrel{=}\Varid{linearD}\;\Varid{addC}{}\<[E]%
\\
\>[B]{}\hsindent{3}{}\<[3]%
\>[3]{}\Varid{mulC}{}\<[9]%
\>[9]{}\mathrel{=}\Conid{D}\;(\Varid{mulC}\mathbin{\smalltriangleup}\lambda (\Varid{a},\Varid{b})\to (\lambda (\Varid{da},\Varid{db})\to \Varid{b}\mathbin{*}\Varid{da}\mathbin{+}\Varid{a}\mathbin{*}\Varid{db})){}\<[E]%
\ColumnHook
\end{hscode}\resethooks
}

\framet{Running examples}{

\begin{hscode}\SaveRestoreHook
\column{B}{@{}>{\hspre}l<{\hspost}@{}}%
\column{E}{@{}>{\hspre}l<{\hspost}@{}}%
\>[B]{}\Varid{sqr}\mathbin{::}\Conid{Num}\;\Varid{a}\Rightarrow \Varid{a}\to \Varid{a}{}\<[E]%
\\
\>[B]{}\Varid{sqr}\;\Varid{a}\mathrel{=}\Varid{a}\mathbin{*}\Varid{a}{}\<[E]%
\\[\blanklineskip]%
\>[B]{}\Varid{magSqr}\mathbin{::}\Conid{Num}\;\Varid{a}\Rightarrow \Varid{a} \times \Varid{a}\to \Varid{a}{}\<[E]%
\\
\>[B]{}\Varid{magSqr}\;(\Varid{a},\Varid{b})\mathrel{=}\Varid{sqr}\;\Varid{a}\mathbin{+}\Varid{sqr}\;\Varid{b}{}\<[E]%
\\[\blanklineskip]%
\>[B]{}\Varid{cosSinProd}\mathbin{::}\Conid{Floating}\;\Varid{a}\Rightarrow \Varid{a} \times \Varid{a}\to \Varid{a} \times \Varid{a}{}\<[E]%
\\
\>[B]{}\Varid{cosSinProd}\;(\Varid{x},\Varid{y})\mathrel{=}(\Varid{cos}\;\Varid{z},\Varid{sin}\;\Varid{z})\;\mathbf{where}\;\Varid{z}\mathrel{=}\Varid{x}\mathbin{*}\Varid{y}{}\<[E]%
\ColumnHook
\end{hscode}\resethooks

In categorical vocabulary:

\begin{hscode}\SaveRestoreHook
\column{B}{@{}>{\hspre}l<{\hspost}@{}}%
\column{E}{@{}>{\hspre}l<{\hspost}@{}}%
\>[B]{}\Varid{sqr}\mathrel{=}\Varid{mulC}\hsdot{\circ }{.}(\Varid{id}\mathbin{\smalltriangleup}\Varid{id}){}\<[E]%
\\[\blanklineskip]%
\>[B]{}\Varid{magSqr}\mathrel{=}\Varid{addC}\hsdot{\circ }{.}(\Varid{mulC}\hsdot{\circ }{.}(\Varid{exl}\mathbin{\smalltriangleup}\Varid{exl})\mathbin{\smalltriangleup}\Varid{mulC}\hsdot{\circ }{.}(\Varid{exr}\mathbin{\smalltriangleup}\Varid{exr})){}\<[E]%
\\[\blanklineskip]%
\>[B]{}\Varid{cosSinProd}\mathrel{=}(\Varid{cosC}\mathbin{\smalltriangleup}\Varid{sinC})\hsdot{\circ }{.}\Varid{mulC}{}\<[E]%
\ColumnHook
\end{hscode}\resethooks
}

\framet{Computation graphs --- example}{
\begin{hscode}\SaveRestoreHook
\column{B}{@{}>{\hspre}l<{\hspost}@{}}%
\column{3}{@{}>{\hspre}l<{\hspost}@{}}%
\column{E}{@{}>{\hspre}l<{\hspost}@{}}%
\>[3]{}\Varid{magSqr}\;(\Varid{a},\Varid{b})\mathrel{=}\Varid{sqr}\;\Varid{a}\mathbin{+}\Varid{sqr}\;\Varid{b}{}\<[E]%
\\
\>[3]{}{}{}\<[E]%
\\
\>[3]{}\Varid{magSqr}\mathrel{=}\Varid{addC}\hsdot{\circ }{.}(\Varid{mulC}\hsdot{\circ }{.}(\Varid{exl}\mathbin{\smalltriangleup}\Varid{exl})\mathbin{\smalltriangleup}\Varid{mulC}\hsdot{\circ }{.}(\Varid{exr}\mathbin{\smalltriangleup}\Varid{exr})){}\<[E]%
\ColumnHook
\end{hscode}\resethooks
\vspace{-5ex}
\begin{center}\wpicture{4.5in}{magSqr}\end{center}

Auto-generated from Haskell code.
See \href{http://conal.net/papers/compiling-to-categories/}{\emph{Compiling to categories}}.
}

\framet{AD example --- \ensuremath{\Varid{sqr}}}{
\vspace{-4ex}
\begin{hscode}\SaveRestoreHook
\column{B}{@{}>{\hspre}l<{\hspost}@{}}%
\column{3}{@{}>{\hspre}l<{\hspost}@{}}%
\column{E}{@{}>{\hspre}l<{\hspost}@{}}%
\>[3]{}\Varid{sqr}\;\Varid{a}\mathrel{=}\Varid{a}\mathbin{*}\Varid{a}{}\<[E]%
\ColumnHook
\end{hscode}\resethooks
\begin{textblock}{160}[1,0](357,37)
\begin{tcolorbox}
\wpicture{2in}{sqr}
\end{tcolorbox}
\end{textblock}
\pause

\vspace{4ex}
\begin{center}\wpicture{4.8in}{sqr-adf}\end{center}

%% \figoneW{0.51}{cosSinProd-ad}{|andDeriv cosSinProd|}{\incpic{cosSinProd-ad}}}
}

\framet{AD example --- \ensuremath{\Varid{magSqr}}}{
\vspace{4ex}
\begin{hscode}\SaveRestoreHook
\column{B}{@{}>{\hspre}l<{\hspost}@{}}%
\column{3}{@{}>{\hspre}l<{\hspost}@{}}%
\column{E}{@{}>{\hspre}l<{\hspost}@{}}%
\>[3]{}\Varid{magSqr}\;(\Varid{a},\Varid{b})\mathrel{=}\Varid{sqr}\;\Varid{a}\mathbin{+}\Varid{sqr}\;\Varid{b}{}\<[E]%
\ColumnHook
\end{hscode}\resethooks
\begin{textblock}{160}[1,0](357,37)
\begin{tcolorbox}
\wpicture{2in}{magSqr}
\end{tcolorbox}
\end{textblock}
\pause

\vspace{4ex}
\begin{center}\wpicture{4.8in}{magSqr-adf}\end{center}

%% \figoneW{0.51}{cosSinProd-ad}{|andDeriv cosSinProd|}{\incpic{cosSinProd-ad}}}
}


\end{document}
